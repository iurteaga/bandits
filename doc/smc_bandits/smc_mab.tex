% !TEX root = smc_bandits.tex
We use sequential Monte Carlo to compute posteriors and sufficient statistics of interest for a rich-class of MABs:
non-stationary bandits (modeled via linear dynamical systems)
with complex (stateless and context-dependent) nonlinear reward functions,
subject to non-Gaussian stochastic innovations.

We model non-stationary, stochastic MABs in a state-space framework,
where for a given reward distribution $p_{a}(Y|x,\theta)$,
and parameters that evolve in-time via a transition distribution $p(\theta_{t}|\theta_{t-1})$,
we write
\begin{equation}
\begin{cases}
\theta_{t}^* \sim p(\theta_{t}^*|\theta_{t-1}^*) \; ,\\
Y_{t}\sim p_{a_t}(Y|x_t,\theta_{t}^*) \; ,
\end{cases}
\quad t=1, \cdots, T,
\label{eq:dynamic_mab}
\end{equation}
where we explicitly indicate with $\theta_t^*$ the true yet unknown parameters of the non-stationary multi-armed bandit.

Within this bandit framework, a Bayesian policy must characterize
the posterior of the unknown parameters $p(\theta_t|\HH_{1:t})$ as in Equation~\eqref{eq:mab_param_posterior},
in which the time-varying dynamics of the underlying transition distribution are incorporated.

The posterior of interest given observed reward $y_t$
can be written as
\begin{equation}
p(\theta_{t}|\HH_{1:t}) \propto p_{a_t}(y_t|x_t, \theta_{t}) p(\theta_{t} |\HH_{1:t-1}) \; ,
\label{eq:dynamic_posterior}
\end{equation}
where $p(\theta_t| \HH_{1:t-1}) = \int_{\theta_{t-1}} p(\theta_{t} | \theta_{t-1}) p(\theta_{t-1}|\HH_{1:t-1}) \dd{\theta_{t-1}}$.
%
Recall that the parameter predictive distribution $p(\theta_t| \HH_{1:t-1})$ and parameter posterior $p(\theta_{t}|\HH_{1:t})$ in Equation~\eqref{eq:dynamic_posterior} 
have analytical, closed-form recursive solutions only for limited cases.

We adhere to the standard MAB formulation,
in that each arm of the bandit is described by its own idiosyncratic parameters
(no information is shared across arms);
\ie $p_{a}(Y| x_t,\theta_{t}^*)=p_{a}(Y|x_t,\theta_{t,a}^*)$,
yet we allow for such parameters to evolve independently per-arm in time:
$p(\theta_{t}^*|\theta_{t-1}^*)=\prod_{a=1}^{|\A|} p(\theta_{t,a}^*|\theta_{t-1,a}^*)$.
Therefore, the posterior of interest factorizes across arms
\begin{equation}
	p(\theta_{t}|\HH_{1:t}) = \prod_{a=1}^{|\A|} p(\theta_{t,a}|\HH_{1:t})\; .
	\label{eq:dynamic_posterior_factorized}
\end{equation}

This standard MAB formulation with independent, per-arm parameter dynamics
enables Bayesian MAB policies to approximate each per-arm parameter posterior separately.
Given observed reward $y_t$ for played MAB arm $a_t$ at time instant $t$,
only the parameter posterior of the played arm is updated with this new observation,
\ie 
\begin{equation}
	p(\theta_{t,a_t}|\HH_{1:t}) \propto p_{a_t}(y_t|x_t, \theta_{t,a_t}) p(\theta_{t,a_t} | \theta_{t-1,a_t}) p(\theta_{t-1,a_t}|\HH_{1:t-1}) \; ;
	\label{eq:dynamic_posterior_factorized_aplayed_updated}
\end{equation}
while parameter posteriors of the non-played arms are only updated according to the latent parameter dynamics, \ie
\begin{equation}
	p(\theta_{t,a}|\HH_{1:t}) \propto p(\theta_{t,a} | \theta_{t-1,a}) p(\theta_{t-1,a}|\HH_{1:t-1}) \;, \forall a^\prime \neq a_t \;.
	\label{eq:dynamic_posterior_factorized_updated}
\end{equation}

We here combine SMC with Bayesian bandit policies ---Thompson sampling and Bayes-UCB, specifically---
for non-stationary bandits as modeled in Equation~\eqref{eq:dynamic_mab}.
The challenge is on computing the posteriors in Equations~\eqref{eq:dynamic_posterior}, \eqref{eq:dynamic_posterior_factorized_aplayed_updated} and \eqref{eq:dynamic_posterior_factorized_updated} for a variety of MAB models,
for which SMC enables us to accommodate 
($i$) any likelihood function that is computable up to a proportionality constant, and
($ii$) any time-varying model described by a transition density from which we can draw samples.

We use SMC to compute per-arm parameter posteriors at each bandit round,
\ie we approximate per-arm filtering densities $p(\theta_{t,a}|\HH_{1:t})$ with SMC-based random measures $p_M(\theta_{t,a}|\HH_{1:t})$, $\forall a$,
for which there are strong theoretical convergence guarantees~\citep{j-Crisan2002,j-Chopin2004}.

The dimensionality of this estimation problem depends on the size of per-arm parameters,
and not on the number of bandit arms $|\A|$.
Consequently, there will be no particle degeneracy due to increased number of arms.

We present in Section~\ref{ssec:sir-policies} and Algorithm~\ref{alg:sir-mab}
the SMC-based Bayesian MAB framework we devise for nonlinear and non-stationary bandits.
%For its implementation, 
%the likelihood function must be computable up to a proportionality constant,
%and one needs to be able to draw samples from the transition density of the parameter dynamics.
We describe in Section~\ref{ssec:linear_mixing_dynamics}
how to draw samples from the transition densities,
when modeling bandit non-stationarity via the general linear model;
and in Section~\ref{ssec:mab_reward_models},
we present examples of
non-Gaussian and nonlinear (continuous and discrete)
reward functions of interest in practice.
Throughout, we avoid assumptions on model parameter knowledge and resort to their Bayesian marginalization.

\subsection{SMC-based Bayesian MAB policies}
\label{ssec:sir-policies}

We combine SMC with both Thompson sampling and Bayes-UCB policies, 
by sequentially updating, at each bandit interaction $t$,
a SMC-based random measure to approximate the time-varying posterior of interest,
\begin{equation}
p(\theta_{t,a}|\HH_{1:t})\approx p_M(\theta_{t,a}|\HH_{1:t})=\sum_{m=1}^M w_{t,a}^{(m)} \delta\left(\theta_{t,a}^{(m)}-\theta_{t,a}\right) \;.
\end{equation}
Knowledge of $p_M(\theta_{t,a}|\HH_{1:t})$ enables computation
of any per-arm reward statistic Bayesian MAB policies require.

We present Algorithm~\ref{alg:sir-mab}
with the sequential Importance Resampling (SIR) method\footnote{
	We acknowledge that any of the methodological SMC advancements that improve and extend SIR,
	\eg advanced SMC algorithms~\citep{ip-Merwe2001, j-Andrieu2010}
	or alternative resampling mechanisms~\citep{j-Li2015},
	are readily applicable to the proposed SMC-based bandit framework,
	and are likely to have a positive impact on the corresponding SMC-based MAB policies' performance.
} 
as introduced by \citet{j-Gordon1993}, 
where:
\begin{itemize}
	\item The SMC proposal distribution $q(\cdot)$ at each bandit interaction $t$ obeys the assumed parameter dynamics:
		$\theta_{t,a}^{(m)} \sim p(\theta_{t,a}|\theta_{t-1,a}), \ \forall m$ ---Step (9.b) in Algorithm~\ref{alg:sir-mab};
	\item SMC weights are updated based on the likelihood of the observed rewards:
		$w_{t,a}^{(m)} \propto p_a(y_t|x_t,\theta_{t,a}^{(m)})$ ---Step (9.c) in Algorithm~\ref{alg:sir-mab}; and
	\item The SMC random measure is resampled at every time instant ---Step (9.a).
\end{itemize}

Independently of which SMC technique is used to compute the posterior random measure $p_M(\theta_{t,a}|\HH_{1:t})$,
the fundamental operation in the proposed SMC-based MAB Algorithm~\ref{alg:sir-mab} 
is to sequentially update the random measure $p_M(\theta_{t,a}|\HH_{1:t})$
to approximate the true per-arm posterior $p(\theta_{t,a}|\HH_{1:t})$
over bandit interactions.

This SMC-based random measure is key,
along with transition density $p(\theta_{t,a}|\theta_{t-1,a})$,
to sequentially propagate parameter posteriors per-arm,
and to estimate their sufficient statistics for any Bayesian bandit policy.
More precisely:
\begin{itemize}
	\item In Step 5 of Algorithm~\ref{alg:sir-mab},
	we estimate the predictive posterior of per-arm parameters,
	as a mixture of the transition densities conditioned on previous samples from $p_M(\theta_{t,a}|\HH_{1:t})$, 
	\begin{equation}
		p_M(\theta_{t+1,a}|\HH_{1:t}) = \sum_{m_{t,a}=1}^{M} w_{t,a}^{(m_{t,a})} p(\theta_{t+1,a}|\theta_{t,a}^{(m_{t,a})}) \; , \; m_{t,a}=1,\cdots, M, \; \forall a\in \A \; .
		\label{eq:smc_parameter_predictive_posterior}
	\end{equation}
	
	\item In Step 9 of Algorithm~\ref{alg:sir-mab},
	we propagate forward the sequential random measure $p_M(\theta_{t,a}|\HH_{1:t})$
	by drawing new samples from the transition density, conditioned on \textit{resampled} particles, \ie
	\begin{equation}
		\theta_{t+1,a}^{(m_{t+1,a})} \sim p(\theta_{t+1,a}|\overline{\theta}_{t,a}^{(m_{t+1,a})}) \; , \; m_{t+1,a}=1,\cdots, M, \; \forall a\in \A \; .
	\end{equation}
\end{itemize}
In both cases, one draws with replacement according to the importance weights in $p_M(\theta_{t,a}|\HH_{1:t})$,
\ie from a categorical distribution with per-sample probabilities $w_{t,a}^{(m)}$: $m_{t,a}^\prime \sim \Cat{w_{t,a}^{(m)}}$.

We now describe in detail how to use the SMC-based posterior random measure $p_M(\theta_{t+1,a}|\HH_{1:t})$ for both Thompson sampling and Bayes-UCB policies:
\ie which are the specific instructions to execute in steps 5 and 7 of Algorithm~\ref{alg:sir-mab}.
\begin{itemize}
	\item \textbf{SMC-based Thompson Sampling:}
	TS operates by drawing a sample parameter $\theta_{t+1}^{(s)}$ from its updated posterior $p(\theta_{t+1}|\HH_{1:t})$,
	and picking the optimal arm for such sample, \ie
	\begin{equation}
		a_{t+1}=\argmax_{a^\prime \in \A} \mu_{t+1,a^\prime}\left(x_{t+1},\theta_{t+1,a^\prime}^{(s)}\right) \; .
		\label{eq:mc_expected_reward}
	\end{equation}
	We use the SMC random measure $p_M(\theta_t|\HH_{1:t})$,
	and propagate it using the transition density $p(\theta_{t+1,a}|\theta_{t,a})$,
	to draw samples from the parameter posterior predictive distribution:
	\ie $\theta_{t+1}^{(s)}\sim p_M(\theta_{t+1}|\HH_{1:t})$
	in Equation~\eqref{eq:smc_parameter_predictive_posterior}.
	This SMC-based random measure provides an accurate approximation to the true posterior density with high probability.

	\vspace*{1ex}	
	\item \textbf{SMC-based Bayes-UCB:}
	We extend Bayes-UCB to reward models where the quantile functions are not analytically tractable,
	by leveraging the SMC-based parameter predictive posterior random measure  $p_M(\theta_{t+1}|\HH_{1:t})$.

	We compute the quantile function of interest by first 
	evaluating the expected reward at each round $t$ based on the available posterior samples,
	\ie $\mu_{t+1,a}^{(m)}\left(x_{t+1},\theta_{t+1,a}^{(m)}\right)$, $m=1,\cdots,M$;
	and compute
	$
	\Prob{\mu_{t+1,a}>q_{t+1,a}(\alpha_{t+1})} = \alpha_{t+1}
	$
	via
	\begin{equation}
		q_{t+1,a}(\alpha_{t+1}):=\max \left\{\mu \; \left|\sum_{m|\mu_{t+1,a}^m>\mu} w_{t,a}^m\ge\alpha_{t+1} \right. \right\} \; .
		\label{eq:mc_quantile_value}
	\end{equation}
	The convergence of quantile estimators generated by SMC methods has been explicitly proved in~\citep{j-Maiz2012}.
\end{itemize}

Random measure $p_M(\theta_{t+1,a}|\HH_{1:t})$ in Equation~\eqref{eq:smc_parameter_predictive_posterior} 
enables computation of the statistics Bayesian MAB policies require,
extending their applicability from stationary to time-evolving bandits.
The exposition that follows addresses dynamic bandits,
and we illustrate how to process classic, stationary bandits within the proposed framework in Appendix~\ref{asec:static_bandits}.

%Algorithm: MAB-SIR
\begin{algorithm}
	\caption{SMC-based Bayesian MAB policies}
	\label{alg:sir-mab}
	\begin{algorithmic}[1]
		\REQUIRE $\A$, $p(\theta_a)$, $p(\theta_{t,a}|\theta_{t-1,a})$, $p_a(Y|x,\theta)$, $\forall a \in \A$.
		\REQUIRE Number of SMC samples $M$ (for UCB we also require $\alpha_t$)
		\STATE Draw initial samples from the parameter prior
			\vspace*{-1ex}
		\begin{equation}
			\overline{\theta}_{0,a}^{(m_{0,a})} \sim p(\theta_a), \quad \text{ and } \quad w_{0,a}^{^{(m_{0,a})}}=\frac{1}{M} \;, \; m_{0,a}=1,\cdots, M, \; \forall a \in \A \;.
			\nonumber
			\vspace*{-4ex}
		\end{equation}
		\FOR{$t=0, \cdots, T-1$}
		\STATE Receive context $x_{t+1}$
		\FOR{$a \in \A$}
		\STATE Estimate sufficient statistics of the MAB policy for all arms, \\
		given $\{w_{t,a}^{(m_{t,a})} \}$ and $\{\theta_{t,a}^{(m_{t,a})}\}$, $\forall m_{t,a}$, $\forall a\in\A$.\\
		
		\vspace*{1ex}
		\quad For \textit{Thompson sampling:}\\
		\qquad Draw a sample $s \sim \Cat{w_{t,a}^{(m_{t,a})}}$, \\
		\qquad Propagate the sample parameter $\theta_{t+1,a}^{(s)}\sim p\left(\theta_{t+1,a}|\theta_{t,a}^{(s)}\right)$, \\
		\qquad Set $\mu_{t+1,a}\left(x_{t+1}, \theta_{t+1,a}^{(s)}\right)=\eValue{}{Y|a,x_{t+1}, \theta_{t+1,a}^{(s)}}$ .\\
		
		\vspace*{1ex}
		\quad For\textit{Bayes-UCB:}\\
		\qquad Draw $M$ candidate samples $m_{a}^\prime \sim \Cat{w_{t,a}^{(m_{t,a})}}$,\\
		\qquad Propagate sample parameters $\theta_{t+1,a}^{(m_{a}^\prime)} \sim p\left(\theta_{t+1,a}|\theta_{t,a}^{(m_{a}^\prime)}\right)$, \\
		\qquad Set $\mu_{t+1,a}\left(x_{t+1}, \theta_{t+1,a}^{(m_{a}^\prime)}\right)=\eValue{}{Y|a,x_{t+1}, \theta_{t+1,a}^{(m_{a}^\prime)}}$,\\
		\qquad Estimate quantile $q_{t+1,a}(\alpha_{t+1})$ as in Equation~\eqref{eq:mc_quantile_value}.
		
		\vspace*{1ex}
		\ENDFOR
		\STATE Decide next action $a_{t+1}$ to play\\
		\vspace*{1ex}
		\quad For \textit{Thompson sampling:} \hspace*{0.6cm} $a_{t+1}=\argmax_{a^\prime \in \A} \mu_{t+1,a^\prime}\left(x_{t+1}, \theta_{t+1,a^\prime}^{(s)}\right)$ \\
		\vspace*{1ex}
		\quad For\textit{Bayes-UCB:} \hspace*{1.8cm} $a_{t+1}=\argmax_{a^\prime \in \A}q_{t+1,a^\prime}(\alpha_{t+1})$
		
		\vspace*{1ex}
		\STATE Observe reward $y_{t+1}$ for played arm
		
		\vspace*{1ex}
		\STATE Update the posterior SMC random measure $p_M(\theta_{t,a}|\HH_{1:t})$ for all arms\\
		
		\begin{enumerate}[(a)]
			\vspace*{-1ex}
			\item Resample $m_{t+1,a}=1,\cdots, M$ parameters $\overline{\theta}_{t,a}^{(m_{t+1,a})}=\theta_{t,a}^{(m_{t,a}^\prime)}$ per arm $a\in \A$,
			where $m_{t,a}^\prime$ is drawn with replacement according to the importance weights $w_{t,a}^{(m_{t,a})}$.

			\vspace*{-1ex}
			\item Propagate resampled parameters by drawing from the transition density
			\vspace*{-1ex}
			\begin{equation}		
				\theta_{t+1,a}^{(m_{t+1,a})} \sim p\left(\theta_{t+1,a} \middle| \overline{\theta}_{t,a}^{(m_{t+1,a})}\right) \; , \; m_{t+1,a}=1,\cdots, M, \; \forall a \in \A \; .
				\label{eq:sir-mab-propagate}
			\vspace*{-2ex}
			\end{equation}
			
			\vspace*{-1ex}
			\item Weight samples of the played arm $a_{t+1}$ based on the likelihood of observed $y_{t+1}$
			\vspace*{-1ex}
			\begin{equation}
				\widetilde{w}_{t+1,a_{t+1}}^{\left(m_{t+1,a_{t+1}}\right)} \propto p_{a_{t+1}}\left(y_{t+1} \middle|  x_{t+1},\theta_{t+1,a_{t+1}}^{\left(m_{t+1,a_{t+1}}\right)}\right) \; ,
				\label{eq:sir-mab-weights}
			\vspace*{-2ex}
			\end{equation}
			and normalize the weights
			\vspace*{-1ex}
			\begin{equation}
				w_{t+1,a_{t+1}}^{\left(m_{t+1,a_{t+1}}\right)}=\frac{\widetilde{w}_{t+1,a_{t+1}}^{\left(m_{t+1,a_{t+1}}\right)}}{\sum_{m_{t+1,a_{t+1}}=1}^M\widetilde{w}_{t+1,a_{t+1}}^{\left(m_{t+1,a_{t+1}}\right)}} \; , \; m_{t+1,a}=1,\cdots, M.
				\label{eq:sir-mab-weights-norm}
			\vspace*{-2ex}
			\end{equation}
		\end{enumerate}
		\vspace*{-2ex}
		\ENDFOR
	\end{algorithmic}
\end{algorithm}

\subsection{Non-stationary MABs}
\label{ssec:linear_mixing_dynamics}
% !TEX root = smc_bandits.tex

The dynamic linear model is a flexible and widely used framework to characterize time-evolving systems~\citep{b-Whittle1951, b-Box1976, b-Brockwell1991, b-Durbin2001, b-Shumway2010, b-Durbin2012}.
Here, we model the latent parameters of the bandit $\theta \in \Real^{d_{\Theta}}$ to evolve over time according to
\begin{equation}
\theta_{t,a}=L_a \theta_{t-1,a}+\epsilon_a \;, \qquad \epsilon_a\sim\N{\epsilon_a|0, \Sigma_a} \; ,
\label{eq:linear_mixing_dynamics}
\end{equation}
with parameters $L_a \in \Real^{d_{\Theta_a} \times d_{\Theta_a}}$ and $\Sigma_a \in \Real^{d_{\Theta_a} \times d_{\Theta_a}}$
---recall that we specify distinct transition densities per-arm.

With \emph{known parameters} $L_a$ and $\Sigma_a$, the transition distribution $p(\theta_{t,a}|\theta_{t-1,a})$
is Gaussian with closed-form updates,
\ie $\theta_{t,a}\sim \N{\theta_{t,a}|L_a \theta_{t-1,a}, \Sigma_a} $.

For the more interesting case of \emph{unknown parameters},
we marginalize parameters $L_a$ and $\Sigma_a$ of the transition distributions
utilized by the proposed SMC-based Bayesian policies, \ie
we Rao-Blackwellize\footnote{
	Rao-Blackwellization is known to help reduce the degeneracy and variance of SMC estimates~\citep{ip-Doucet2000,ib-Djuric2010}.
} them.

The marginalized transition density is a multivariate t-distribution~\citep{j-Geisser1963,j-Tiao1964,j-Geisser1965,j-Urteaga2016,j-Urteaga2016a}:
\begin{align}
\theta_{t,a} \sim \T{\theta_{t,a}|\nu_{t,a}, m_{t,a}, R_{t,a}} \;, \text{ with } \quad
\begin{cases}
\nu_{t,a}=\nu_{0,a}+t-d \; ,\\
m_{t,a}=L_{t-1,a} \theta_{t-1,a} \; , \\
R_{t,a} = \frac{V_{t-1,a}}{\nu_{t,a}\left(1-\theta_{t-1,a}^\top(U_{t,a} U_{t,a}^\top)^{-1}\theta_{t-1,a}\right)} \; ,\\
\end{cases} 
\label{eqn:dynamics_rb} \\
\nonumber \\
\text{where } \begin{cases}
\Theta_{t_0:t_1,a}=[\theta_{t_0,a} \theta_{t_0+1,a} \cdots \theta_{t_1-1,a} \theta_{t_1,a}] \in \Real^{d\times (t_1-t_0)} \; , \\
B_{t-1,a} = \left(\Theta_{0:t-2,a}\Theta_{0:t-2,a}^\top + B_{0,a}^{-1} \right)^{-1} \; ,\\
L_{t-1,a} = \left(\Theta_{1:t-1,a}\Theta_{0:t-2,a}^\top + L_{0,a}B_{0,a}^{-1}\right) B_{t-1,a} \; ,\\
V_{t-1,a}= \left(\Theta_{1:t-1,a}-L_{t-1,a} \Theta_{0:t-2,a}\right)\left(\Theta_{1:t-1,a}-L_{t-1,a} \Theta_{0:t-2,a}\right)^\top \\
\qquad \qquad + \left(L_{t-1,a}-L_{0,a}\right) B_{0,a}^{-1} \left(L_{t-1,a}-L_{0,a}\right)^\top + V_{0,a} \; ,\\
U_{t,a} U_{t,a}^\top = \left(\theta_{t-1,a}\theta_{t-1,a}^\top+B_{t-1,a}^{-1}\right) \; .\\
\end{cases}
\label{eqn:dynamics_rb_aux}
\end{align}
Each distribution above holds separately for each arm $a$, and subscript $_{a,0}$ indicates assumed prior parameters for arm $a$.

These transition distributions are used when propagating per-arm parameter densities in Steps 5 and 9 of Algorithm~\ref{alg:sir-mab}.
They are fundamental for the accuracy of the sequential, random measure-based approximation to the posterior,
and the downstream performance of the proposed SMC-based MAB policies.

Caution must be exercised when using SMC to approximate the dynamic bandit model's posteriors.
%On the one hand, certain parameter constraints might be necessary for the model to be wide-sense stationary~\citep{b-Box1976, b-Shumway2010, b-Durbin2012}.
Notably, the impact of non-Markovian transition distributions in SMC performance must be taken into consideration:
the sufficient statistics in Equations~\eqref{eqn:dynamics_rb}-\eqref{eqn:dynamics_rb_aux} depend on the full history of the model dynamics.
Here, we use general linear models,
for which it can be shown that, if stationarity conditions are met,
the autocovariance function decays quickly,
\ie the dependence of general linear models on past samples decays exponentially~\citep{j-Urteaga2016,j-Urteaga2016a}.

When exponential forgetting holds in the latent space
---\ie the dependence on past samples decays exponentially, and is negligible after a certain lag---
one can establish uniform-in-time convergence of SMC methods for functions that depend only on recent states, see~\citep{j-Kantas2015} and references therein.

More broadly, one can establish uniform-in-time convergence for path functionals that depend only on recent states,
as the Monte Carlo error of $p_M(\theta_{t-\tau:t}|\HH_{1:t})$ with respect to $p(\theta_{t-\tau:t}|\HH_{1:t})$ is uniformly bounded over time.
This quick forgetting property is fundamental
for the successful performance of SMC methods for inference of linear dynamical states in practice~\citep{j-Urteaga2017b,j-Urteaga2016,j-Urteaga2016a}.

Nevertheless, we acknowledge that any improved SMC solution that mitigates path-degeneracy issues can only be beneficial for the performance of the proposed SMC-based policies.

\subsection{MAB reward models}
\label{ssec:mab_reward_models}
% !TEX root = smc_bandits.tex

Algorithm~\ref{alg:sir-mab} is described in terms of a generic reward likelihood function $p_a(Y|x_t,\theta_{t,a})$
that must be computable up to a proportionality constant.
We now introduce reward functions that are applicable in many MAB use-cases,
where the time subscript $_t$ has been suppressed for clarity of presentation,
and subscript $_0$ indicates assumed prior parameters.

\subsubsection{Contextual, categorical rewards}
\label{sssec:categorical_softmax_rewards}

For MAB problems where observed rewards are discrete,
\ie $Y=c$ for $c\in\{1,\cdots,C\}$,
and contextual information is available,
the softmax function is a natural reward density model.
In general, categorical variables assign probabilities to an unordered set of outcomes ---not necessarily numeric.
In this work,
we refer to categorical rewards where,
for each categorical outcome $c\in\Natural$,
there is a numeric reward $y=c$ associated with it.

Given a $d$-dimensional context vector $x\in\Real^{d}$,
and per-arm parameters $\theta_{a,c} \in \Real^{d}$ for each category $c\in\{1,\cdots,C\}$,
the contextual softmax reward model is
\begin{equation}
p_a(Y=c|x,\theta_a)=\frac{e^{(x^\top\theta_{a,c})}}{\sum_{c'=1}^C e^{(x^\top\theta_{a,c'})} } \; ,
%p_a(y=c|x,\theta_a)=\exp{(x^\top\theta_{a,c})}/\sum_{c'=1}^C\exp{(x^\top\theta_{a,c'})} \; .
\label{eq:softmax_rewards}
\end{equation}
where we denote with $\theta_a=\{\theta_{a,1}, \cdots, \theta_{a,C}\}$ the set of per-category parameters $\theta_{a,c}$ for arm $a$.
For this reward distribution,
the posterior of the parameters can not be computed in closed form,
and neither, the quantile function of the expected rewards $\mu_{t,a}=y_t\cdot(x_t^\top\theta_{t,a})$.

When returns are binary, \ie $Y=\{0,1\}$ (success or failure of an action),
but dependent on a $d$-dimensional context vector $x\in\Real^{d}$,
the softmax function reduces to the logistic reward model
\begin{equation}
p_a(Y|x,\theta)=\frac{e^{Y\cdot(x^\top\theta_a) }}{1+e^{(x^\top\theta_a)}} \; ,
\label{eq:logistic_rewards}
\end{equation}
with per-arm parameters $\theta_a \in \Real^{d}$ of same dimensionality $d$ as the context $x$.

The theoretical study of UCB and TS-based algorithms for logistic rewards is an active research area~\citep{ip-Dong2019,ip-Faury2020},
which we here extend to the discrete-categorical setting.
%
We accommodate discrete-categorical MAB problems by implementing Algorithm~\ref{alg:sir-mab}
with likelihoods as in Equations~\eqref{eq:softmax_rewards} or~\eqref{eq:logistic_rewards}.
Namely, we compute $p_M(\theta_{t,a}|\HH_{1:t})$ 
for both stationary and non-stationary discrete-categorical bandits,
by updating the weights of the posterior SMC random measure in Step 9-c.
To the best of our knowledge,
no existing work addresses non-stationary, discrete-categorical bandits.

\subsubsection{Contextual, linear Gaussian rewards}
\label{sssec:linear_gaussian_rewards}

For bandits with continuous rewards, Gaussian distributions are typically used,
where contextual dependencies can easily be included.
The contextual linear Gaussian reward model is well studied in the bandit literature~\citep{ic-Abbasi-Yadkori2011,ip-Chu2011,ip-Agrawal2013a},
where the expected reward of each arm is modeled as a linear combination of a $d$-dimensional context vector $x\in\Real^{d}$,
and the idiosyncratic parameters of the arm $w_a\in\Real^{d}$; \ie
\begin{equation}
\begin{split}
p_a(Y|x,\theta)&=\N{Y \middle|x^\top w_a, \sigma_a^2} =\frac{1}{\sqrt{2\pi\sigma_a^2}}e^{-\frac{(Y-x^\top w_a)^2}{2\sigma_a^2}} \; .
\end{split}
\end{equation}
We denote with $\theta\equiv\{w, \sigma\}$ the set of all parameters of the reward distribution,
and consider the normal inverse-gamma conjugate prior distribution for these, 
\begin{equation}
\begin{split}
p(w_a, \sigma_a^2|u_{0,a}, V_{0,a}, \alpha_{0,a}, \beta_{0,a})
%&= \NIG{w_a, \sigma_a^2|u_{0,a}, V_{0,a},\alpha_{0,a}, \beta_{0,a}} \\
& = \N{w_a|u_{0,a}, \sigma_a^2 V_{0,a}} \cdot \IG{\sigma_a^2|\alpha_{0,a}, \beta_{0,a}} \; .\\
%& = \frac{e^{-\frac{1}{2}(w_a-u_{0,a})^\top(\sigma_a^2 V_{0,a})^{-1}(w_a-u_{0,a})}}{(2\pi)^{1/2}\sigma_a \mydet{V_{0,a}}^{-1/2}} \cdot \frac{\beta_0^{\alpha_0}}{\Gamma\left(\alpha_0\right)} (\sigma_a^2)^{-\alpha_0-1}e^{-\frac{\beta_0}{(\sigma_a^2)}} \; .
\end{split}
\end{equation}

After observing actions $a_{1:t}$ and rewards $y_{1:t}$, the parameter posterior for each arm 
\begin{equation}
\begin{split}
p(w_a, \sigma_a^2|a_{1:t},y_{1:t},u_{0,a}, V_{0,a},\alpha_{0,a}, \beta_{0,a}) &= p\left(w_a, \sigma_a^2|u_{t,a}, V_{t,a},\alpha_{t,a}, \beta_{t,a}\right) \\
%&=\NIG{w_a, \sigma_a^2|u_{t,a}, V_{t,a},\alpha_{t,a}, \beta_{t,a}} \; ,
\end{split}
\end{equation}
follows an updated normal inverse-gamma distribution with sequentially updated hyperparameters
\begin{equation}
\begin{cases}
V_{t,a}^{-1} = V_{t-1,a}^{-1} + x_t x_t^\top \cdot \mathds{1}[a_t=a] \; ,\\
u_{t,a}= V_{t,a} \left( V_{t-1,a}^{-1} u_{t-1,a} + x_t y_{t}\cdot \mathds{1}[a_t=a]\right) \; ,\\
\alpha_{t,a}=\alpha_{t-1,a} + \frac{\mathds{1}[a_t=a]}{2} \; ,\\
\beta_{t,a}=\beta_{t-1,a} + \frac{\mathds{1}[a_t=a](y_{t_a}-x_t^\top u_{t-1,a})^2}{2\left(1+x_t^\top V_{t-1,a} x_t\right)} \; ,
\end{cases}
\end{equation}
or, alternatively, batch updates of the form
\begin{equation}
\begin{cases}
V_{t,a}^{-1}= V_{0,a}^{-1}+x_{{1:t}|t_a} x_{{1:t}|t_a}^\top \; ,\\
u_{t,a}=V_{t,a}\left(V_{0,a}^{-1}u_{0,a}+x_{{1:t}|t_a} y_{{1:t}|t_a}\right) \; ,\\
\alpha_{t,a}=\alpha_{0,a} + \frac{|t_a|}{2} \; ,\\
\beta_{t,a}=\beta_{0,a} + \frac{\left(y_{{1:t}|t_a}^\top y_{{1:t}|t_a} + u_{0,a}^\top V_{0,a}^{-1}u_{0,a} - u_{t,a}^\top V_{t,a}^{-1}u_{t,a} \right)}{2} \; ,
\end{cases}
\end{equation}
where $t_a=\{t|a_t=a\}$ indicates the set of time instances when arm $a$ is played.

With these, we can compute the Bayesian expected reward of each arm,
\begin{equation}
p(\mu_{a}|x, \sigma_a^2, u_{t,a}, V_{t,a}) = \N{\mu_{a} \middle|x^\top u_{t,a}, \; \sigma_a^2 \cdot x^\top V_{t,a} x} \; ,
\label{eq:gaussian_posterior_mean}
\end{equation}
and the quantile function for such distribution
\begin{equation}
q_{t+1,a}(\alpha_{t+1})=Q\left(1-\alpha_{t+1}, \N{\mu_{a} \middle|x^\top u_{t,a}, \; \sigma_a^2 \cdot x^\top V_{t,a} x}\right) \;.
\label{eq:gaussian_posterior_quantile}
\end{equation}

The reward variance $\sigma^2_a$ is unknown in practice,
so we marginalize it and obtain
\begin{equation}
\begin{split}
p(\mu_{a}|x, u_{t,a}, V_{t,a}) = \T{\mu_{a} \middle|2\alpha_{t,a}, x^\top u_{t,a}, \; \frac{\beta_{t,a}}{\alpha_{t,a}} \cdot x^\top V_{t,a} x} \;, 
%& \qquad = \frac{\Gamma\left(\frac{2\alpha_{t,a}+1}{2}\right)}{\Gamma\left(\frac{2\alpha_{t,a}}{2}\right)\sqrt{\pi 2\alpha_{t,a} \frac{\beta_{t,a}}{\alpha_{t,a}} x^\top V_{t,a} x}} \cdot \left(1+\frac{1}{(2\alpha_{t,a})}\left(\frac{(\mu_a-x^\top u_{t,a})^2}{\frac{\beta_{t,a}}{\alpha_{t,a}} \cdot x^\top V_{t,a} x}\right)\right)^{-\frac{2\alpha_{t,a}+1}{2}} \; .
\end{split}
\label{eq:t_posterior_mean}
\end{equation}
which leads to quantile function computations based on a Student's t-distribution
\begin{equation}
q_{t+1,a}(\alpha_{t+1})=Q\left(1-\alpha_{t+1}, \T{\mu_{a} \middle| 2\alpha_{t,a}, x^\top u_{t,a}, \; \frac{\beta_{t,a}}{\alpha_{t,a}} \cdot x^\top V_{t,a} x}\right) \;.
\label{eq:t_posterior_quantile}
\end{equation}

Equations~\eqref{eq:gaussian_posterior_mean} and~\eqref{eq:gaussian_posterior_quantile}
are needed in Step 5 when implementing TS or Bayes-UCB policies for the known $\sigma_a^2$ case;
while Equations~\eqref{eq:t_posterior_mean} and~\eqref{eq:t_posterior_quantile} are used for the unknown $\sigma_a^2$ case.
Note that one can use the above results for Gaussian bandits with no context, by replacing $x=I$ and obtaining $\mu_{a}=u_{t,a}$.

When these equations are combined
with the Rao-Blackwellized transition densities derived for the dynamic model in Section~\ref{ssec:linear_mixing_dynamics},
the proposed SMC-based MAB policies can be applied to non-stationary, linear Gaussian bandit problems with minimal assumptions:
\ie only the functional form of the transition and reward functions is known.

There are no competing MAB algorithms
for non-stationary bandit problems where no parameter knowledge is assumed.

