% !TEX root = smc_bandits.tex

The dynamic linear model is a flexible and widely used framework to characterize time-evolving systems~\citep{b-Whittle1951, b-Box1976, b-Brockwell1991, b-Durbin2001, b-Shumway2010, b-Durbin2012}.
Here, we model the latent parameters of the bandit $\theta \in \Real^{d_{\Theta}}$ to evolve over time according to
\begin{equation}
\theta_{t,a}=L_a \theta_{t-1,a}+\epsilon_a \;, \qquad \epsilon_a\sim\N{\epsilon_a|0, \Sigma_a} \; ,
\label{eq:linear_mixing_dynamics}
\end{equation}
with parameters $L_a \in \Real^{d_{\Theta_a} \times d_{\Theta_a}}$ and $\Sigma_a \in \Real^{d_{\Theta_a} \times d_{\Theta_a}}$
---recall that we specify distinct transition densities per-arm.

With \emph{known parameters} $L_a$ and $\Sigma_a$, the transition distribution $p(\theta_{t,a}|\theta_{t-1,a})$
is Gaussian with closed-form updates,
\ie $\theta_{t,a}\sim \N{\theta_{t,a}|L_a \theta_{t-1,a}, \Sigma_a} $.

For the more interesting case of \emph{unknown parameters},
we marginalize parameters $L_a$ and $\Sigma_a$ of the transition distributions
utilized by the proposed SMC-based Bayesian policies, \ie
we Rao-Blackwellize\footnote{
	Rao-Blackwellization is known to help reduce the degeneracy and variance of SMC estimates~\citep{ip-Doucet2000,ib-Djuric2010}.
} them.

The marginalized transition density is a multivariate t-distribution~\citep{j-Geisser1963,j-Tiao1964,j-Geisser1965,j-Urteaga2016,j-Urteaga2016a}:
\begin{align}
\theta_{t,a} \sim \T{\theta_{t,a}|\nu_{t,a}, m_{t,a}, R_{t,a}} \;, \text{ with } \quad
\begin{cases}
\nu_{t,a}=\nu_{0,a}+t-d \; ,\\
m_{t,a}=L_{t-1,a} \theta_{t-1,a} \; , \\
R_{t,a} = \frac{V_{t-1,a}}{\nu_{t,a}\left(1-\theta_{t-1,a}^\top(U_{t,a} U_{t,a}^\top)^{-1}\theta_{t-1,a}\right)} \; ,\\
\end{cases} 
\label{eqn:dynamics_rb} \\
\nonumber \\
\text{where } \begin{cases}
\Theta_{t_0:t_1,a}=[\theta_{t_0,a} \theta_{t_0+1,a} \cdots \theta_{t_1-1,a} \theta_{t_1,a}] \in \Real^{d\times (t_1-t_0)} \; , \\
B_{t-1,a} = \left(\Theta_{0:t-2,a}\Theta_{0:t-2,a}^\top + B_{0,a}^{-1} \right)^{-1} \; ,\\
L_{t-1,a} = \left(\Theta_{1:t-1,a}\Theta_{0:t-2,a}^\top + L_{0,a}B_{0,a}^{-1}\right) B_{t-1,a} \; ,\\
V_{t-1,a}= \left(\Theta_{1:t-1,a}-L_{t-1,a} \Theta_{0:t-2,a}\right)\left(\Theta_{1:t-1,a}-L_{t-1,a} \Theta_{0:t-2,a}\right)^\top \\
\qquad \qquad + \left(L_{t-1,a}-L_{0,a}\right) B_{0,a}^{-1} \left(L_{t-1,a}-L_{0,a}\right)^\top + V_{0,a} \; ,\\
U_{t,a} U_{t,a}^\top = \left(\theta_{t-1,a}\theta_{t-1,a}^\top+B_{t-1,a}^{-1}\right) \; .\\
\end{cases}
\label{eqn:dynamics_rb_aux}
\end{align}
Each distribution above holds separately for each arm $a$, and subscript $_{a,0}$ indicates assumed prior parameters for arm $a$.

These transition distributions are used when propagating per-arm parameter densities in Steps 5 and 9 of Algorithm~\ref{alg:sir-mab}.
They are fundamental for the accuracy of the sequential, random measure-based approximation to the posterior,
and the downstream performance of the proposed SMC-based MAB policies.

Caution must be exercised when using SMC to approximate the dynamic bandit model's posteriors.
%On the one hand, certain parameter constraints might be necessary for the model to be wide-sense stationary~\citep{b-Box1976, b-Shumway2010, b-Durbin2012}.
Notably, the impact of non-Markovian transition distributions in SMC performance must be taken into consideration:
the sufficient statistics in Equations~\eqref{eqn:dynamics_rb}-\eqref{eqn:dynamics_rb_aux} depend on the full history of the model dynamics.
Here, we use general linear models,
for which it can be shown that, if stationarity conditions are met,
the autocovariance function decays quickly,
\ie the dependence of general linear models on past samples decays exponentially~\citep{j-Urteaga2016,j-Urteaga2016a}.

When exponential forgetting holds in the latent space
---\ie the dependence on past samples decays exponentially, and is negligible after a certain lag---
one can establish uniform-in-time convergence of SMC methods for functions that depend only on recent states, see~\citep{j-Kantas2015} and references therein.

More broadly, one can establish uniform-in-time convergence for path functionals that depend only on recent states,
as the Monte Carlo error of $p_M(\theta_{t-\tau:t}|\HH_{1:t})$ with respect to $p(\theta_{t-\tau:t}|\HH_{1:t})$ is uniformly bounded over time.
This quick forgetting property is fundamental
for the successful performance of SMC methods for inference of linear dynamical states in practice~\citep{j-Urteaga2017b,j-Urteaga2016,j-Urteaga2016a}.

Nevertheless, we acknowledge that any improved SMC solution that mitigates path-degeneracy issues can only be beneficial for the performance of the proposed SMC-based policies.